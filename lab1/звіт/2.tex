\chapter{Алгоритм розв'язання}


Зробимо крайові умови однорідними шляхом заміни $u(x) = w(x) + v(x)$, 
де $v(x)$ будемо шукати з системи: 
$$
\begin{cases}
    v(x) = rx + l \\
    -k(a) v^{\prime}(a)+\alpha_1 v(a)=\mu_1 \\
    k(b) v^{\prime}(b)+\alpha_2 v(b)=\mu_2
\end{cases}
$$

Після підстановки:

$$
\begin{cases}
    r(-k(a) +\alpha_1 a) + \alpha_2 l =\mu_1 \\
    r(k(b) +\alpha_2 b) + \alpha_2 l =\mu_2
\end{cases}
$$


За теоремою Крамера: 



$$r = \frac{\begin{vmatrix}
            \mu_1 & \alpha_1 \\
            \mu_2 & \alpha_2
        \end{vmatrix}}{
        \begin{vmatrix}
            (-k(a) +\alpha_1 a) & \alpha_1 \\
            (k(b) +\alpha_2 b) & \alpha_2
        \end{vmatrix}},
l = -\frac{\begin{vmatrix}
            (-k(a) +\alpha_1 a) & \mu_1 \\
            (k(b) +\alpha_2 b) & \mu_2
        \end{vmatrix}}{
        \begin{vmatrix}
            (-k(a) +\alpha_1 a) & \alpha_1 \\
            (k(b) +\alpha_2 b) & \alpha_2 
        \end{vmatrix}}
$$

\pagebreak
Таким чином 


$$
\begin{cases}
    r = \frac{\mu_1\alpha_2 - \mu_2\alpha_1}{ -k(a)\alpha_2 + \alpha_1\alpha_2a - \alpha_1k(b) - \alpha_1\alpha_2b}\\ 
    l = \frac{\mu_2k(a) - \mu_2\alpha_1a + \mu_1k(b) +\mu_1\alpha_2b}{-k(a)\alpha_2 + \alpha_1\alpha_2a - \alpha_1k(b) - \alpha_1\alpha_2b}
\end{cases}
$$


Підставимо отриману $v(x)$ у головне рівняння:
$$-(k(x) w^{\prime})^{\prime}+p(x) w^{\prime}+q(x) w - rk'(x) + rp(x) + q(x)rx + q(x)l = f(x)$$
$$-(k(x) w^{\prime})^{\prime}+p(x) w^{\prime}+q(x) w = rk'(x) - rp(x) - q(x)rx - q(x)l + f(x) = f_1(x)$$


Нова задача з однорідними умовами:


$$-(k(x) w^{\prime})^{\prime}+p(x) w^{\prime}+q(x) w=f_1(x), a<x<b,$$ 
$$-k(a) w^{\prime}(a)+\alpha_1 w(a)=0,$$
$$ k(b) w^{\prime}(b)+\alpha_2 w(b)=0,$$


Знаходимо базисні функції $\varphi_1, \varphi_2, ... \varphi_n$ такі, що виконують граничні умови.


$$
\begin{cases}
    \varphi_1 = (x - a)^2 (x - C) \\ 
    \varphi_2 = (b - x)^2 (x - D) \\
    \varphi_i = (x - a)^2 (b - x)^{i-1}, i \geq 3 
\end{cases} 
$$


Тоді наша функція буде знаходитись у вигляді $w(x) = \sum\limits_{i=1}^{n} c_i \varphi_i $
\pagebreak

Для методу Бубнова-Гальоркіна значення коефіцієнтів $c_i$ можна розрахувати із СЛАР
$$\sum_{i=1}^{n} c_i (A\varphi_i, \varphi_\gamma) = (f, \varphi_\gamma), \gamma = 1, ... n$$


Для методу найменших квадратів 


$$ \sum_{i=1}^{n} c_i (A\varphi_i, A\varphi_\gamma) = (f, A\varphi_\gamma), \gamma = 1...n $$


Кінцеву похибку обчислимо за формулою:


$$\varepsilon = \| u - u_n \| = \sqrt{\frac{1}{b-a} \int_{a}^{b} (u - u_n)^2dx}$$
