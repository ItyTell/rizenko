\chapter{Висновки}

\begin{enumerate}
    \item     Зі збільшенням кількості векторів для наближеного розв'язку точність покращується. Це підтверджує важливість вибору оптимальної кількості векторів для досягнення необхідної точності.

   \item Метод Бубнова-Гальоркіна виявився більш точним в порівнянні з методом найменших квадратів, коли використовується та ж сама кількість векторів. Це може свідчити про його більшу ефективність для дослідження даних граничних задач.

    \item З результатів видно, що при використанні 7 векторів точність наближеного розв'язку стає дуже високою, що робить його дуже корисним для практичного використання в розв'язанні граничних задач.
\end{enumerate}