\chapter{Постановка задачі}

Методи побудови різницевих схем

Інтегро-інтерполяційний метод.

Розглянемо наступну граничну задачу:
$$
\left\{\begin{array}{l}
-\left(k u^{\prime}\right)^{\prime}+q u=f, 0<x<1, \\
-k u^{\prime}+\alpha_1 u=\mu_1, \quad x=0, \\
k u^{\prime}+\alpha_2 u=\mu_2, \quad x=1,
\end{array}\right.
$$

де $k=k(x) \geq c_0>0, q=q(x) \geq 0, \quad k(x) \in Q^1[a, b], q(x), f(x) \in Q[a, b]$.
Побудуємо сітку $\bar{\omega}_h=\left\{x_i=i h, h=1 / N, i=\overline{0, N}\right\}$ та проінтегруємо (1) по відрізку $\left[x_{i-1 / 2}, x_{i+1 / 2}\right]$, отримаємо :
$$
w_{i+1 / 2}-w_{i-1 / 2}+\int_{x_{i-1 / 2}}^{x_{i+1 / 2}}(q u-f) d x=0,
$$

де $w_i=-k\left(x_i\right) \frac{d u}{d x}\left(x_i\right)$ - величина потоку, який проходить через точку $x_i$.

Розглянемо
$$
\int_{x_{i-1 / 2}}^{x_{i+1 / 2}} q(x) u(x) d x=u(\xi) \int_{x_{i-1 / 2}}^{x_{i+1 / 2}} q(x) d x \approx u_i \int_{x_{i-1 / 2}}^{x_{i+1 / 2}} q(x) d x=h d_i u_i,
$$

де $\xi \in[a, b]$,
$$
d_i=\frac{1}{h} \int_{x_{i-1 / 2}}^{x_{i+1}} q(x) d x
$$

Аналогічно
$$
\int_{x_{i-1 / 2}}^{x_{i+1}} f(x) d x=\varphi_i h,
$$

де
$$
\varphi_i=\frac{1}{h} \int_{x_{i-1 / 2}}^{x_{i+1 / 2}} f(x) d x .
$$

Проінтегруємо по відрізку $\left[x_{i-1}, x_i\right]$ вираз $u^{\prime} =-\frac{w(x)}{k(x)}$. Отримаємо
$$
u_i-u_{i-1}=-\int_{x_{i-1}}^{x_i} \frac{w(x)}{k(x)} d x \approx-w_{i-1 / 2} \int_{x_{i-1}}^{x_i} \frac{1}{k(x)} d x,
$$

або
$$
w_{i-1 / 2} \approx-a_i u_{\bar{x}, i},
$$

де


$$
a_i=\left(\frac{1}{h} \int_{x_{i-1}}^{x_i} \frac{1}{k(x)} d x\right)^{-1}
$$

Враховуючи також вираз
$$
w_{i+1 / 2} \approx-a_{i+1} u_{\overline{x}, i+1}
$$
отримаємо однорідну різницеву задачу
$$
-\left(a y_{\bar{x}}\right)_{x, i}+d_i y_i=\varphi_i, i=\overline{1, N-1}
$$

Проінтегруємо співвідношення (1) по відрізку $\left[0, x_{1 / 2}\right]$ :
$$
w_{1 / 2}-w_0+\int_0^{x_1 / 2}(q u-f) d x=0 .
$$
$$
w_0=\mu_1-\alpha_1 u_0,
$$

а при $i=1$
$$
w_{1 / 2} \approx-a_1 u_{x, 0} .
$$

Скористаємося також наближеннями
$$
\int_0^x \frac{1}{2} q u d x \approx u_0 \int_0^x \frac{1}{2} q(x) d x=\frac{h}{2} u_0 d_0,
$$

де


$$
d_0=\frac{2}{h} \int_0^{x_{1 / 2}} q(x) d x,
$$

та
$$
\int_0^{x_{1 / 2}} f(x) d x=\frac{h}{2} \varphi_0,
$$

де
$$
\varphi_0=\frac{2}{h} \int_0^{x_{1 / 2}} f(x) d x
$$

отримуємо різницеву апроксимацію граничних умов :
$$
-a_1 y_{x, 0}+\alpha_1 y_0-\mu_1+\frac{h}{2} d_0 y_0-\frac{h}{2} \varphi_0=0,
$$

або
$$
-a_1 y_{x, 0}+\overline{\alpha_1} y_0=\overline{\mu_1},
$$

де
$$
\overline{\alpha_1}=\alpha_1+\frac{h}{2} d_0, \overline{\mu_1}=\mu_1+\frac{h}{2} \varphi_0 .
$$